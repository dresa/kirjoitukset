\documentclass[a4paper, 12pt, finnish]{article}
\usepackage{babel}
\usepackage[utf8]{inputenc}
\usepackage[T1]{fontenc}
\usepackage{amsmath} %align
\usepackage[all]{nowidow}  % no widow/orphan lines

\title{Mielenliikutus}
\author{Dresa}
\date{\today}

% quotation, lainaus: \q{Häh}, sanoin.
\newcommand{\q}[1]{''#1''}

% jump to the next scene, kohtauksen vaihto, aikaa kuluu
\def\jump{\vspace{2mm} \centerline{$\diamondsuit$} \vspace{2mm}}
% mark ending
\def\endjump{\vspace{2mm} \centerline{$\diamondsuit\diamondsuit\diamondsuit$} \vspace{2mm}}
% alternatives: \textbullet


\begin{document}

\maketitle

\q{Hienoa! Saadaan sitten perusteellisempi henkilökuva}, vastasi Taina
innokkaasti, kun annoin hänelle toissapäivänä luvan tonkia
taustatietojani. Lupa oli tuskin muuta kuin muodollisuus,
mutta jos kerran haastatellaan ammattilaisten voimin,
niin tehdään se sitten hänen tavallaan, ajattelin. 


\jump


Tapaamispäivänä merkitsin, tapojeni mukaisesti, viestimeeni
määränpään ja reittisuunnitelmani, vaikka oltiinkin samassa kaupungissa
ja tunsin kohteen.
Taksi odotti jo pihalla, kun pääsin ulos pilvisen taivaan alle.

\q{Täytyy varmaan ottaa takapenkkikin käyttöön}, tokaisi taksikuski,
kun näki minun raahaavan säkkikaupalla tavaraa ulko-ovelta.

\q{Minne vaan mahtuu}, huudahdin, \q{Ei tarvitse turhaan varoa.}

Pyysin kuskia viemään minut ensin läheiseen konditoriaan,
jonka antimista olin nautiskellut jo lukemattomia kertoja aiemmin.
Huusin konditorian ovelta saman tien: \q{Neljä munkkipossua, kiitos!}

Halusin olla omavarainen makean suhteen. Jos vaikka toimituksessa ei ymmärrettäisi makean päälle.

Maksoin ostoksesta kymmenkertaisesti, mitä uusi myyjä jäi kummastelemaan.
Tällaiset odottamattomat hyvät eleet saivat tyypillisesti
muut hämmennyksen valtaan, mutta eiväthän he ymmärtäneet,
että raha oli jo menettänyt merkityksensä --- ainakin minulle.

Nousin takaisin taksiin. Kirjoitin munkkikääreen päälle \q{haastatteluun},
ja otin kääreestä ja sen sisällöstä kuvan vaistomaisesti. Annoin
kuskille uuden osoitteen, ja aloin taitella munkkikuitista origamia.


\jump


Taksi johdatti minut lopulta toimituksen pääovelle. Kuskin kasvoilta
heijastui jälleen tuttu ällistys, kun halusin maksaa matkasta
kymmenkertaisesti. En vieläkään ollut keksinyt pettämätöntä tapaa
vakuuttaa muut vilpittömyydestäni. Lopulta rahat kelpasivat kuskille,
joka mumisi jotakin epäkiinnostavaa maksun verotuksesta.
 
Nousin ulos autosta ja katselin ympärilleni. Kadulla näkyi vain
satunnaisia uneliaita ohikulkijoita, joita saapumiseni ei näyttänyt
tippaakaan kiinnostavan. Ja miksi olisikaan. Kuski ehti jo avata
tavarasäilön luukun, kun pyysin häntä odottamaan. Olin nimittäin
pyytänyt jo etukäteen Tainaa järjestämään kantoapua --- kysyminen ei
maksa mitään.

Noudatin saamaani ohjeistusta: marssin pääovesta sisään ja
ilmoittauduin vastaanottotiskillä. Palvelu oli huvittavan kömpelöä,
ja ihmekös tuo, kun päätin puhua huvin vuoksi
vain urdua.

Muutaman minuutin kuluttua Taina saapui vastaanottoaulaan,
ja hänen seurassaan pari toimittajaharjoittelijaa.
Taina ja minä esittäydyimme toisillemme. Tapasin Tainan nyt ensi kerran. Arvioin hänet viisikymppiseksi, ja ensivaikutelmani hänestä oli jäykän virallinen.
Vastaanottovirkailija katseli minua palvelutiskin takaa kummasti,
kun äkkiä puhuinkin täysin ymmärrettävällä kielellä.

\q{Aineistoni on vielä taksissa}, kerroin. Anonyymit harjoittelijat ymmärsivät
heti yskän ja lähtivät noutamaan tavaraa autosta.

\q{Vien teidät kokoustilaan}, sanoi Taina. \q{Tavaranne saapuvat kyllä pian.}


\jump


Toimitus teki työtään modernissa sokkelossa, jossa hissit ja
avainkortit paljastivat uusia piilotettuja käytäviä. Siellä täällä
oli toisistaan erotettuja osastoja, joissa harhaili uneliaita
olentoja, joiden työtehtäviä tuskin kukaan ymmärsi.

Tainan mukaan olimme perillä. Meitä odotti kahdenkymmenen vuoden
takaista tyyliä edustava kokoustila.

\q{Onko huone sopiva teille?}

\q{On kyllä}, totesin. Huone oli jopa suurempi kuin olin pyytänyt.
Valtava pöytä jakoi tilan, ja seinät oli koristeltu muinaisilla
presidenteillä. Kokoustilan vieressä oli reilunkokoinen
edustusparveke ja käytävällä pieni kahvila, josta sai muun
muassa munkkipossuja.

\q{Odotellessamme, että avustajani tuovat tavaranne,
haluaisitteko kenties kahvilasta jotakin?}

Kohteliaisuussyistä päätin
jättää aiemmin hankkimani munkit syrjään: \q{Iso kuppi mustaa kahvia
ja munkkipossu}, vastasin.

Palasimme Tainan kanssa kokoushuoneeseen ja asetuimme aloillemme,
pyynnöstäni pöydän vastakkaisiin päätyihin.
Jos toiveeni tuntuikin Tainasta oudolta, en havainnut sitä hänestä.
Pilvistä säätä koskeva kankea sananvaihtomme muuttui lopulta
virikkeelliseksi keskusteluksi hyvän munkkipossun määritelmästä.
Hyvä aihe.

Kun avustajat olivat kantaneet tavarani viereeni ja poistuneet
ilmeettöminä, Taina sulki oven ja aloitti:
\q{Mikäli teille sopii, niin aloitan kertomalla lyhyesti,
miksi pidän haastatteluanne erityisen kiinnostavana. Sen jälkeen
voisimme käydä vapaata keskustelua ja pitää taukoja silloin kun haluatte.
Arvioni mukaan tässä menee pari tuntia. Ajattelin myös tallentaa
haastattelun. Sopiiko tällainen järjestely teille?}

Minulla ei ollut mitään esitettyä vastaan.

\q{Kiinnostukseni perustuu onneen, joka teitä on elämässänne kohdannut.
Mitä enemmän saan elämästänne selville, sitä onnekkaammalta vaikutatte.
Toinen syy on siinä, miten jokapäiväistä elämäänne luonnehditaan.}

\q{Mitä elämästäni on kerrottu?} kysyin kiinnostuneena.

\q{Merkitsette kuulemma muistiin elämänne kaikkein pienimmätkin
yksityiskohdat, sekä menneet että tulevat, onko näin?}

\q{Kyllä vain, tallennan elämääni muun muassa tekstin, äänen,
kuvan ja videon keinoin. Tekninen kehitys tuo jatkuvasti
uusia keinoja.}

Avasin avustajien tuomat säkit ja kaivoin esille sekalaisia
papereita, muistivihkoja, tallenteita, viestimiä
ja soitinlaitteita vuosien takaa.

\q{Tässä näette ensimmäiset merkintäni, jotka tein 22-vuotiaana},
sanoin, ja vein kasan muistivihkoja ja laitteita Tainan
luokse. \q{Tapani ovat muuttuneet noista ajoista huomattavasti:
ajatusten tallentaminen ja niiden palauttaminen mieleen on
nykytekniikalla paljon helpompaa.}

Taina selaili merkintöjäni hetken. Yksittäisten bussimatkojen,
ruokailusuunnitelmien ja kunnallisvaalitulosten tarkka raportoiminen
ei vaikuttanut saavan osakseen paljon ymmärrystä, jos lainkaan.

Kokoustilan seinältä minua tuijotti taulustaan presidentti,
jonka kasvoja en tunnistanut.

\q{Voitte pitää kaiken tänne
tuomani aineiston, jos vain haluatte}, totesin --- eihän minulla
ollut enää käyttöä merkinnöilleni.

Taina ei vastannut mitään.

\q{Mitä varten tallennatte elämäänne?}, kysyi Taina hetken kuluttua.
Halusin vältellä aihetta vielä hetken, jotta saisin kuulla,
mikä oli saanut Tainan ottamaan minuun yhteyttä
toissapäivänä.

Vastasin: \q{Minulle on tärkeää tuntea menneisyys.
Lupaan vastata aiheeseen pian.}

Taina jatkoi taustojen kartoittamista: \q{Päädyitte julkisuuteen
sen jälkeen, kun useiden vedonlyöntitoimistojen palvelimille
tehtiin laaja tietomurto. Vaikka palvelimilta kerätty data olikin
hieman kyseenalaista, tai ehkä juuri siksi, monet uhkapeleistä
kiinnostuneet halusivat löytää ne pelaajat, jotka olivat osuneet
useimmin oikeaan.
Nuo tilastollisen analyysin tulokset julkaistiin kolme päivää
sitten PeliLeaks-verkkosivulla. Oletteko tutustunut tuloksiin?}

\q{En ole}, vastasin. Se oli totta, mutta arvasin kyllä tulosten merkityksen.

\q{Teidän nimenne on joka tapauksessa jokaisella listalla kärkipäässä.
Tuolloin ajattelin, että olette vain taitava vedonlyöjä, ja olette
saanut voittonne ahkeralla taustatyöllä. Sen jälkeen hankkimani
tiedon valossa vaikuttaa kuitenkin siltä, että taito ei yksin riitä
selitykseksi. Missä vain liikuttekin, teille tuntuu siunaantuvan kohtuullisia 
arvontavoittoja ja sijoitustuottoja samalla kun epäonni karttaa teitä.

Kaikesta päätellen pidätte matalaa profiilia, mutta pienistä
puroista on lopulta kertynyt suuria jokia, ja olette nyt täällä.
Mistä on kyse?}

Minulla oli jo käsitys, että jättiotsikot saanut tietomurto
olisi pääsyy kiinnostukseen minua kohtaan. Nyt sain varmistuksen
tärkeimmästä tekijästä, joka johti viime päivien
julkisuusongelmiini: ilman epäonnista tietomurtoa jäljilleni ei olisi
päästy helposti. Halusin vielä tuntea tuttavapiirini
juorunvälittäjät, joten heitin vastakysymyksen:
\q{Ennen kuin vastaan, haluaisin vielä tietää,
mistä muualta kuin PeliLeaks-sivulta olette saanut tietonne?}

\q{Ymmärrättehän, etten voi kertoa moista}, hän vastusteli.

Vastauksen sävystä päättelin, etten saisi haluamaani vastausta,
joten annoin asian olla.
Olihan minulla joka tapauksessa ennakkokäsitys näistä auliista
tietolähteistä. Päätin, etten enää olisi niin höveli heidän kanssaan.

Ehdotin Tainalle lyhyttä taukoa: halusin käydä parvekkeella. Samaa
juhlallista parveketta olin joskus ihaillut katutasolta
ja miettinyt, tuuleeko siellä kovaakin. Katutasolla ei ainakaan
tuullut tänään lainkaan.

Parvekkeen lattia oli kauttaaltaan märkä edellisen yön sateiden
jäljiltä. Yläkerroksissa tunsi jo kevyestäkin tuulesta, kuinka
kosteaa ilma oli. Söin viimeisetkin munkkipossusta.


\jump


Samalla kun siirryimme takaisin sisälle, pyysin lasillisen vettä,
jonka Taina toikin heti minulle. Palasimme takaisin pöydän eri
päätyihin, neljän metrin päähän toisistamme.
Jutustelimme hetken kokoushuoneen vanhentuneesta sisustuksesta,
mutta juttu ei oikein sujunut. Etäisyys välillämme ei tainnut helpottaa.

\q{Hyvä on. Kerron teille onneni lähteen}, julistin.
\q{Pystyn matkaamaan ajassa.}

Pidin tässä kohtaa
tauon, jonka aikana tarkkailin Tainaa kiinnostuneena.
Kerroin nimittäin kyvystäni ensi kertaa. Taina tarkasteli
minua vähäeleisen asiallisesti. Piti minua luultavasti hulluna!
Halusin antaa itsestäni kuvan järkevänä miehenä, joten
jatkoin selitystäni: \q{Pidän kirjaa kaikista tapahtumista,
jotta pystyisin muistamaan menneisyyden tapahtumat riittävän
tarkasti. Vietän elämääni kuten haluan ja tarvittaessa
palaan menneisyyteen.}

Taina oli mietteisään, kuten sopii odottaa.

\q{Olette siis matkustanut ajassa paljonkin?} tuumaili Taina nyt
aiempaa varautuneemmin.

\q{Kyllä}, vastasin, \q{vähintään satoja kertoja --- en ole
itse asiassa pitänyt lukua}.

\q{Kertokaa, mitä viestiä tulevaisuus tuo?} uteli Taina.

Päätin jättää ivan huomiotta, ja sanoin:
\q{Itse asiassa en ole koskaan matkustanut
tulevaisuuteen. Kun matkaan menneisyyteen, niin
olen toki menneisyyden näkökulmasta nähnyt joitakin
tulevaisuuden asioita. Siirryn yleensä vain
lähimenneisyyteen. En ole kuitenkaan koskaan elänyt
kalenterin mukaan näin pitkälle, joten minulla
ei ole sen enempää tietoa tulevaisuudesta kuin teilläkään.}

\q{Miten harmillista}, hän pilkkasi.

Kun en reagoinut Tainan kommenttiin millään muotoa, hän jatkoi:
\q{Vaikka ette voi kertoa tulevaisuudesta, pystyttekö jollain muulla
tavalla todistamaan, että pystytte aikamatkailuun?}

Minulla ei tosiaan ollut vakuuttavia todisteita, mutta annoin tapahtuneen
puhua puolestaan. Vastasin: \q{Olen lunastanut esimerkiksi
suurimmat vedonlyöntivoittoni elämällä siihen asti,
kunnes näen tuloksen, painamalla tuloksen mieleeni,
tekemällä lyhyen aikamatkan lähimenneisyyteen,
ja lyömällä vetoa näkemäni tuloksen puolesta. Useimmiten tulos pysyy samana.}

\q{Useimmiten?}, ihmetteli Taina.

\q{Niin}, vastasin. \q{Menneisyydessä käyttäydyin tietyllä tavalla,
ja tästä seurasi näkemäni lopputulos. Kun nykymieleni siirtyy
menneisyyteen, käytökseni yleensä poikkeaa edellisestä kerrasta,
ja tämä voi johtaa erilaiseen lopputulokseen. Kyse voi olla myös
jostakin kvanttimekaniikan ilmiöstä --- en tunne sitä maailmaa.
Oli miten oli, tulevaisuus ei aina toistu samalla tavalla.}

Aihepiirin muuttuminen oli selvästi yllättänyt Tainan.
Hän käytti hetken keksiäkseen mielekkäitä kysymyksiä, ehkäpä
lukemansa science fiction -kirjallisuuden pohjalta. Hän antoi palaa: 
\q{Kuinka nämä aikamatkat oikeastaan tapahtuvat? Pitäisikö minun yllättyä, jos
näkisin teidät kolmessa eri paikassa samaan aikaan? Eikö näistä
tilanteista synny helposti jotakin paradoksaalista?}

\q{Ei sentään}, vastasin nopeasti, \q{kehoani on vain yksi kappale.}
Tämä lause kuulosti oudolta jopa minusta.
\q{Kun teen aikamatkan,
mieleni siirtyy silmänräpäyksessä menneisyyden kehooni. Koska kehossa on
uusi mieli, teen eri asioita kuin edellinen kehossa ollut mieli,
joten aikajatkumo taitaa siinä kohtaa haarautua.}

\q{Näkyykö tämä uusi mieli sitten jollakin fyysisellä tavalla kehossa?} Taina kysyi.

Tuumailin hetken ja vastasin: \q{Totta puhuen, en ole koskaan ymmärtänyt
tai ollut kiinnostunut, mitä siinä tapahtuu. En ole esimerkiksi
lähtenyt vertailemaan magneettikuvia aivoistani. Elämäntapani ei
muuttuisi, oli selitys mikä hyvänsä.}


\jump


Pilvet olivat nyt harvenneet. Sälekaihtimet loivat auringonsäteistä
raidallisen kuvion suurelle pöydälle välissämme.

\q{Eikö aikamatkalle lähteminen aiheuta vaaratilanteita?} kysyi Taina.

\q{Kyllä vain. Kun ensi kertaa kokeilin aikamatkustusta, en tietenkään uskonut
koko asiaan, joten en odottanut siinä olevan mitään riskiä. Siitä lähtien
olen noudattanut tinkimätöntä kurinalaisuutta.}

Hörppäsin tässä kohtaa lasistani kylmää vettä. Dramatiikan mestari.

\q{Ensinnäkin, en ole koskaan
aikamatkustanut kehooni, joka olisi nuorempi kuin 22 vuotta.
En ole varma, pystyykö esimerkiksi lapsen pieni pää tai
vasta kehittymässä olevat aivot ottamaan vastaan nykyistä mieltäni.
Sitä paitsi, teini-iän läpi eläminen uudelleen ei inspiroi minua.}

\q{Sanos muuta}, Taina kommentoi.

\q{Toinen asia on}, jatkoin, \q{että siirryn vain sellaisiin
ajanhetkiin, jotka ovat turvallisia. Merkintäni auttavat minua
löytämään sopivan ajanhetken. Kolmanneksi, tulevaisuus on
epävarma --- myös minulle --- joten
en rohkene siirtyä tulevaisuuteen.} 

\q{Kerroitte käyttävänne merkintöjä ajanhetken valinnassa.
Käytätte paljon vaivaa niiden laatimiseen --- miten merkinnät
auttavat teitä?} Taina kysyi.

\q{Aivan. On katsokaas turvallisempaa siirtyä nukkuvaan kehoon
kuin liikenteen seassa kulkevaan,
sillä ajan ja paikan orientaatio häiriintyy hetkeksi.
Merkintäkokoelmani tarjoaa käyttööni laajan kokoelman
menneisyyden tilanteita ja hetkiä, joista voin valita.

Ennen kuin siirryn menneisyyteen, merkintöjen avulla voin lisäksi opiskella
siihen aikaan liittyviä tietoja: näin en vaikuta liian oudolta.}

\q{Totta puhuakseni}, Taina ehti sanoa, \q{teitä on
kuvailtu\dots\allowbreak omalaatuiseksi, pyrkimyksistänne huolimatta.}

Taina vaikutti nyt keksivän uusia kysymyksiä vaivatta.
\q{Koetteko olevanne etäinen tai hajamielinen?}, kuului seuraava.

\q{Muiden silmissä varmasti molempia}, naurahdin. \q{Oletetaan, että
olen sopinut ystäväni kanssa juhlien järjestämisen yksityiskohdista.
Sitten omasta näkökulmastani elän vaikkapa kolmenkymmenen vuoden
ajan aikamatkustuksen avulla ja palaan jälleen tuohon hetkeen.

Jos muistiinpanoni ovat hatarat, en välttämättä muista juuri
mitään siitä keskustelusta, joka ystäväni mielestä
käytiin hetki sitten. Pakko myöntää, että ihmissuhteeni ovat
kärsineet: elämäni varmaankin näyttää muiden silmissä olevan
vain tyhjän elämän tallentamista.}

Taina tarttui yksityiskohtaan: \q{Käytitte esimerkkinä kolmenkymmenen
vuoden mittaista aikamatkaa. Se oli vain heitto, mutta kuinka
vanha katsotte olevanne? Kehonne on kolmekymmentäviisivuotiaan,
se näkyy henkilötiedoistanne, mutta kuinka vanha mielenne on?}

Jäin pohtimaan asiaa hetkeksi. Aikakäsitykseni poikkesi
ymmärrettävistä syistä tavallisesta, joten en ollut koskaan
tullut ajatelleeksi mieleni ikää. Lopulta vastasin:
\q{Vanhempi kuin kukaan muu, sen verran osaan sanoa.

Olen kolunnut maailman halki,
tutustunut ihmiselämän monipuolisuuteen, opetellut haluamani
taidot, kerännyt lähes rajatta erilaisia kokemuksia
ja nähnyt ajastamme erilaisia versioita. Niin, ja lisäksi
viiniä, laulua ja nautintoja.}

\q{Palaan vielä varsinaiseen kykyynne. Mitä teette,
jotta aikamatka tapahtuu? Siis käytännössä?} Taina vaati saada
tietää.

Sanoin varoen: \q{Ennen kuin vastaan, pyydän teitä
pysymään jatkossa paikallanne. Jos haluatte nähdä lähemmin,
käyttäkää kiikareita, jotka jätin muistikirjojen viereen.}

\q{Tämähän on naurettavaa: katsella nyt kiikareilla pöydän yli!} paheksui Taina.

\q{Aikamatkustuksen avulla pystyn välttämään pienet harmit,
mutta varsinainen ongelma piilee yllättävissä vaaroissa. Haluan
varmistaa, ettette ole vaaraksi. Onko selvä?}
 
\q{Selvä}, hän vastasi ja yllättyi puheeni totisuudesta.


\jump


Kaivoin taskustani nyrkin kokoisen metallisen rasian. Avasin sen
hitaasti niin, että Taina näki sen sisällön. Jos Taina olikin välillä
ollut syvästi mietteissään kaikesta sanotusta, nyt hän oli
pelkkää tarkkaavaisuutta. Hän katseli edessään pitelemääni
pallonmuotoista esinettä, joka sai ilman hohtamaan heikosti
puolen metrin säteellä. Taina nosti kiikarit välillä silmilleen,
kun hän tarkasteli pallon lukemattomia yksityiskohtia.

\q{Tuoko...}, Taina aloitti, mutta ei osannut jatkaa.
Tähän asti hänessä vallinnut rutinoitunut tyyneys
korvautui innokkuudella, jonka pallon läpitunkeva hohde synnytti.

\q{Tämä pallo on aikakoneeni}, aloitin selitykseni.
Isäni serkku, jota pidin viisaana mutta merkillisenä
miehenä, oli määrännyt testamentissaan minulle
lahjoitettavaksi erään tallelokeron sisällön,
kun täyttäisin 22 vuotta. Tallelokerosta
löysin tämän pallomaisen laitteen, erikoisia vieraita
tekstejä ja minulle osoitetun kirjeen.

Kirjeessä kerrottiin, kuinka laite oli aikoinaan löydetty
Länsi-Australiasta aavikon kätkemistä muinaisista raunioista,
ja kuinka laite oli sittemmin päätynyt omistajalta toiselle.

Laitteessa ei ollut minkään nykykulttuurin piirteitä.
Sen pinnalla oli monimutkainen koneisto,
jolla voi asettaa halutun ajanhetken. Oletin,
että laitteen aiemmat omistajat olivat löytäneet hajanaisista
dokumenteista neuvot, joilla he oppivat laitteen käytön.
Itse en osannut tulkita sitä kolmiulotteisia kirjoitusta.

Kerroin Tainalle kaiken, mitä tiesin pallon alkuperästä ja muustakin.
Taina kuunteli tarkkaavaisena, mutta hänen mieleensä tulvi niin paljon
kysymyksiä, ettei osannut kysyä yhtäkään.

\q{Kun käytän laitetta,} jatkoin, \q{se särkyy samalla käyttökelvottomaksi.
Särkyminen on samalla viimeinen asia, jonka näen ennen aikamatkaa.
Kun siirryn menneisyyteen, siellä on aina ehjä laite tallessa,
joten pystyn jatkamaan elämäntapaani. Uskon, että jos siirtyisin
tulevaisuuteen, tämä laite olisi rikki käytön seurauksena.}

Join loput nyt jo auringossa lämmenneestä vedestäni,
mikä varmasti vaikutti Tainasta oudolta tyyneydeltä.
\q{Tämä on ainoa aikakone, jonka tiedän.}

\q{Sen uskon}, Taina vastasi.

Hetken hiljaisuuden jälkeen Taina otti taas ohjat käsiinsä:
\q{Mitä aiotte tehdä seuraavaksi?}

\q{Olen aina pyrkinyt poistamaan kaikki uhat laitteelleni
ja elämäntavalleni. Siksi päätin jo ennen tänne tuloani, että
aion estää vedonlyöntitoimistojen tietomurron. Tuosta
tapahtumasta alkoi julkisuus, joka nyt rajoittaa elämäntapaani.
Palaan noin vuoden verran ajassa taaksepäin.}

\q{Jos kerran haluatte välttää kaikkia riskejä, niin miksi
ihmeessä tulitte tänne haastateltavaksi?} Taina kummasteli.

\q{Niin}, huokaisin. \q{En pitänyt haastattelua kovin vaarallisena,
sillä olenhan varautunut: aikakone on ollut koko ajan käyttövalmiina.
En ole puhunut aikamatkailusta aiemmin 
kenenkään kanssa, ja olen kaivannut ulkopuolisen mielipidettä.
Minua kiinnosti myös tulla haastatelluksi ensi kertaa.

Aikamatkailun tutkiminen on osoittautunut itselleni
liian vaikeaksi, enkä ole voinut ottaa tutkijoihin
yhteyttä laitteen menettämisen pelossa.
Halusin antaa tässä samalla merkittävän jutunaiheen ja
testata elämäntapani järkevyyttä keskustelemalla.

Mitä sanotte? Vaikka elämäntapani onkin vieras,
vaikuttaako toimintani hyvin perustellulta?}

Taina vastasi epäröiden: \q{Tuota, en osaa sanoa.
En ole vielä kunnolla käsittänyt aikakoneen suomia
mahdollisuuksia. Ehkä se on järkevää.}
 

\jump


Keskustelimme vielä hetken vaihtoehtoisista elämäntavoista.
Kerroin, että onnettomuuksien estäminen aikamatkailun
avulla ei kuulunut harrastuksiini. \q{Mielenrauha, julkisuusvaara,
pisara meressä --- tiedättehän.}

Minusta alkoi tuntua, että haastattelun jatkamisesta ei olisi
enää iloa itselleni.

\q{Ehkä on minun aikani poistua. Luvallanne teen aikamatkan tässä
huoneessa. Uskoakseni teille ei koidu sen enempää vaaraa kuin
ympäröivälle maailmallekaan, mutta pyydän silti lupaanne.}

\q{Tottahan toki!} vastasi Taina. \q{Saamme tallennetuksi itse matkankin.}

\q{Olkoon niin}, sanoin. \q{Kun käynnistän laitteen, siitä lähtee
valoa ja se särkyy. Kaikki on ohi sekunnissa, sen verran tiedän.
Sitten voitte nousta tuolilta ja jatkaa työtänne.
Oli hauska tavata. Ehkä käyn tervehtimässä teitä menneisyydessä,
vaikkapa juttuvinkkien varjolla. Hyvästi.}

\q{Hyvästi}, Taina vastasi ristiriitaisin miettein.

Puristin aikakonetta ja liu'utin samalla sormiani pallon pinnalla
tottunein liikkein. Pallon valo alkoi samalla kirkastua ja kaareva
pinta murtui otteessani yhä pienemmiksi sirpaleiksi,
jotka valuivat sormieni välistä. Voimakas valo sattui hetken aikaa
silmiini ja tunsin menettäväni tasapainoni.


\jump


\q{Oletteko kunnossa?} kysyi Taina huolestuneena.

Huomasin makaavani
lattialla, jolle olin ilmeisesti kaatunut. Nousin istumaan, ja
kun katselin nopeasti ympärilleni, huomasin tuntemattoman
presidentin katseen seinällä ja kätkemäni munkkikääreen pöydän alla.
Jalkojeni alla oli kasa teräväreunaisia hiutaleita.
Aurinko paistoi ikkunasta kirkkaasti.

Katsoin Tainaa hämmentyneenä. Lohdutukseksi Taina esitti hetken
kuluttua näkemyksensä: \q{Mietin tässä vain.
Ette maininnut mitään siitä, mitä nykykehollenne käy aikamatkalla.
Ehkä aikakone ei \emph{siirrä} mieliä, kuten väititte,
vaan \emph{kopioi} niitä. Ehkä nykymielenne on nyt kopioitu
vuoden takaiseen vaihtoehtomenneisyyteen.}

Taina käänsi katseensa kahvilaan:
\q{Pysykää hetki paikallanne --- tuon teille lisää vettä.}


\endjump


\end{document}



??? Muutokset:
- kokoushuoneen näkö, esineet
- alkuun lisää "outoa toimintaa"
- parvekkeelle tauolle
- uusi kahvitauko
- lisää persoonallisuutta hahmoihin
- puhetyylit erilaisiksi
- Tainalle lisää roolia
- takauma Peterin kirjeen sisällöstä
- lisää yksityiskohtia
- onko alussa mysteeriä?
- tee ilmaisusta tiiviimpää
- repliikkien merkitsemisen säännöt?
- hahmojen nimet
- nimi
- kappalejako, repliikkien kappalejako
- sukunimet pois, kertoja käyttää etunimiä, hahmot eivät käytä nimiä
- anonyymit avustajat
- esittelyn repliikit pois: sano epäsuorasti

Nimiehdotuksia:
- Kadonnutta aikaa eläessä
- Päähänpisto
- Soita se uudelleen, Sam
- Haluaisin vain, että minua kuunnellaan
- Tuhat ja yksi aikaa/kertaa
- Mielipuoli
- Mielenliikutus

Järjestys:
???Matkalla
??? Saapuminen
??? Esittely
* tavallinen kohtaaminen T:n kanssa
??? Taustatiedot
* julkisuuden tiedot, läheisten haastattelut
??? Paljastus
* 
??? Elämäntavan puolustus
* jokapäiväisen elämän omituisuuksien perustelu
??? Keskustelu (käytäntö)
??? Tauko
??? Keskustelu (filosofia)
Miksi suostuit haastatteluun ja kerroit kaken tämän?
??? Aikamatka



Novelli-idea 11.12.2012, aiheeseen palattu 6.1.2014

== SPOLIERZ ==

trilleri, dekkari, sci-fi, 20 sivua

---

Meta:
- tyylilaji: mysteeri, dekkari, scifi
- aiheet: X:n elämän yksityiskohdat, filosofia, epävarmuus
- neutraali: jokin kaupunki, jokin aika
- imperfekti
- kertojan näkökulma: X (ehkä kaikkitietävä, ei T)

---

Tarinan runko:
- Henkilö X matkustaa tapaamiseen
- Paikalla on toimittaja T
- T haastattelee X:ää oudon elämän ja tapahtumien perusteella
- X paljastaa kykynsä aikamatkustukseen ja perustelee toimintaansa
- T utelee aikamatkustamisen ja X:n elämän yksityiskohdista, tauko
- X käyttää laitettaan 

Loppu:
- X käyttää laitetta ja se särkyy.
- Mitään muuta näkyvää ei tapahdu.
- T arvelee, että aikamatkassa X: mieli ei siirry, vaan _kopioituu_ menneisyyteen eri aikalinjalla
- X on hämmennyksen vallassa (laite rikki, elämäntapa jatkossa mahdoton, elämäntilanne ei paras)

---

Henkilö X:
- X:llä on kyky siirtyä ajassa taaksepäin
- X kuljettaa muistilappuja ja kalenteria mukanaan: lähiaikojen tekemiset. Muuten tulevaisuuden X ei osaisi toimia, jos sattuisi siirtymään tähän menneisyyden aikaan. Aina on paljon kirjaamattomia pikkujuttuja mielessä.
- X ei tiedä, mitä nykykeholle tapahtuu, kun hän siirtyy ajassa.
- Kirjoittaa muistiin tärkeät valinnat ja elämän kulun.
- Valitsee aikamatkustuksen kohdeajan muistioiden perusteella
- X:n ikä ei ole selvillä. Ehkä satoja vuosia.
- X paljastaa aikamatkustuksen ensi kertaa, T:lle.
- X:llä tarve puhua jonkun kanssa aikamatkustuksesta

X:n aikamatkojen hyödyntäminen:
- ei liian räikeitä onnistumisia, mutta toisaalta tylsää tehdä asiat huomaamattoman hitaasti
- aikamatkustus menneisyyteen pitää suurinmman osan asioista ennallaan, mutta hiljalleen asiat alkavat muuttua.
- pidettävä matalaa profiilia, jotta huomio ei kiinnittyisi liiaksi itseen ja muuttuvaan käytökseen.
- X on saanut viime aikoina liikaa julkisuutta holtittomalla käytöksellään 
- ansaitsemiskeinoja:
  * arvonta,
  * uhkapeli,
  * sijoitus,
  * varastaminen,
  * säästöt (maksetaan, koetaan, sitten siirrytään menneisyyteen)
- viini, laulu, naiset
- jos herätetty liikaa huomiota, on parempi kulkea ajassa taaksepäin ja käyttäytyä toisin, jotta epäilykset hälvenevät.
- nyt on X:lle sellainen hetki, joka pitää peruuttaa aikamatkailulla: julkisuus rajoittaa käytöksen muuttamista
- vain lyhyitä aikamatkoja, eikä mitään linjaa ole eletty kovin pitkälle

X:n elämä:
- elämänmuutos 22-vuotiaana, kun sai aikakoneen
- täynnä outojen muistiinpanojen tekemistä
- viime aikojen tekninen kehitys mahdollistaa historian tallentamisen ja haun teksti- ääni-, kuva- ja videomuotoisista historiatallenteista.
- asiat ja muistikuvat sekaantuvat
- heikot sosiaaliset kontaktit (syvälliset ihmissihteet puuttuvat näennäisesti)
- tärkeintä elämässä on välttää onnettomuuksia ja salamurhia yms.
- tekee hyviä tekoja mielenrauhansa vuoksi, vaikka palaakin kohta menneisyyteen

Toimittaja T:
- Toimittaja tekee työtään: juttu tulossa
- Toimittaja kerää X:ltä vastauksia havaittuun outoon käytökseen.
- Kyseenalaistaa X:n toiminnan järkevyyttä

Aikakone:
- kyky on saatu aikaan välineellä
- laite saatu perintönä joltakin omalaatuiselta sukulaiselta?
- laitteen alkuperä lovecraftlaisittain Australiasta
- Laitteita tunnetaan vain yksi.
- "pallomainen, jonka pinnalta säädetään haluttu ajanhetki, vieraissa yksiköissä"
- sukulainen tappoi itsensä laitteen ja ohjeiden luovutuksen jälkeen
- laite särkyy käytettäessä; käyttäjä havaitsee tämän
- laitetta säilytetään varmassa paikassa. Jos laite rikkoontuu, aikamatkustus muuttuu mahdottomaksi.
- T joutuu tarkkailemaan laitetta etäältä, kiikareilla

Aikamatkustuskyky:
- aikamatkustuslaite on kertakäyttöinen (menneisyyden X:llä on aina ehjä laite tallella)
- tulevaisuuteen siirtymistä ei ole tutkittu
- menneisyyteen siirryttäessä laite on vielä ehjä; tulevaisuudessa laite on jo rikkoutunut.
- aikamatkustuksessa X siirtyy menneisyyden kehoonsa, mutta nykyisellä miellellään
- rajoite: aikuisen mieli lapsen aivoihin? Aivot kyvykkäimmillään 23-vuotiaana?
- jokainen aikamatka luo ehkä uuden haaran, johon menneisyys alkaa kulkea?
- kannattaa siirtyä menneisyyteen, kun kohde nukkuu; muuten orientaatio pettää.
- on vaarallista siirtyä aikaan, jossa ei tiedä omaa sijaintiaan (keskellä liikennettä tms.)
- Tiedon siirtäminen menneisyyteen tapahtuu mielessä. Opetellaan ulkoa, siirrytään, kirjoitetaan.
- aikamatkailun fyysinen perusta on vielä tuntematon

Aikamatkustuksen todennettavuus:
- itse aikamatkustus ei ole todennettavissa varmaksi millään keinolla
- tulevaisuuden tapahtumat eivät toteudu aivan suoraan: joko X:n vaikutus tai kvantti-ilmiö
- yksinkertainen magneettikuvatesti ei toimi: kuva pitäisi siirtää myös menneisyyteen vertailun vuoksi.
- magneettikuvatesti:
  * magneettikuva aivoista hetkellä 1,
  * eletään hetkeen 3, otetaan uusi magneettikuva, ja opetellaan erot ekaan kuvaan ulkoa
  * siirrytään menneisyyteen hetkeen 2, merkitään muistiin magneettikuvien erot ajanhetkien 1 ja 3 välillä.
  * otetaan magneettikuva hetkellä 2 ja tunnistetaan erot ekaan kuvaan.
  * jos kuvien erot ovat yhtä, eli hetkien 2 ja 3 kuvat ovat "samat", aikamatkustus muuttaa aivojen rakennetta.
- tutkijoiden mukaanottaminen vaarantaisi laitteen
- ei suurta mielenkiintoa vaikeiden asioiden selvittämiseen
- magneettikuvatestiä ei ole vielä tehty.
- X ei ota kokeilunhalussaan riskejä aikamatkustuksessa.
- Eka aikamatkustuskerta oli riki, mutta silloin hän ei edes uskonut.
- On epäselvää, mitä aikamatkaajan alkuperäiselle keholle käy matkatessa.

Filosofinen näkökulma:
- dualismi: mieli on kehosta erillinen ja voi siirtyä kehosta toiseen ajassa.
- keho on ruokittava, mutta mieli ei varsinaisesti tarvitse mitään välttämätöntä (pitkä ikä)
- suuri harppaus menneisyyteen on myös rasite: elämää on elettävä uusiksi myös tylsistä hetkistä.
- ikuinen nuoruus on mahdollinen, mutta ympäristö toistuu silloin samana.
- ihmiskontaktit vaativat pitkän aikajänteen.
- Uusia hyviä ystäviä ei voi luoda keinotekoisesti aikamatkustuksessa.
- mahdolliset maailmat -näkökulma
- Sattumien käsittely: joissakin mahdollisissa maailmoissa tapahtuu, toisissa ei.

---

Muuta:
- Viittaus Groundhog Dayhin: sama mahdollista jos huvittaa


