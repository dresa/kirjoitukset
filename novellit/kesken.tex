\documentclass[a4paper, 12pt, finnish]{article}
\usepackage{babel}
\usepackage[utf8]{inputenc}
\usepackage[T1]{fontenc}

\usepackage{amsmath} %align
\usepackage[all]{nowidow}  % no widow/orphan lines

\usepackage{refcount}
\usepackage{lettrine}

% lettrine formatting
% https://www.lemald.org/blog/?p=108
\setlength{\DefaultNindent}{0em}
\setcounter{DefaultLines}{4}
\input Kramer.fd % AnnSton, Kramer  % Romantik.fd
\newcommand*\initfamily{\usefont{U}{Kramer}{xl}{n}}
\renewcommand{\LettrineFontHook}{\initfamily}

\newcommand{\reader}[1]{\emph{#1}}
\newcommand{\q}[1]{''#1''}

\def\mytitle{Hätätilanteessa muista sivu \pageref{intermission}}  % päivitä! pageref{intermission}


\title{\mytitle}
\author{Dresa}
\date{\today}


\def\beginning{Olet juuri alkamassa lukea Italo Calvinon uutta romaania
Jos kerran talviyönä matkamies. }


\begin{document}

\maketitle\label{titlekey}

\beginning

\reader{\q{Heh! Hetkinen, tämäpä yllätys! Mikäs tämän tarinan
nimi olikaan?} kummastelet. Tarkistat etusivulta
\pageref{titlekey}, että tarinan nimi on \mytitle,
kuten odotitkin. Aloitat lukemisen taas alusta.}
 
\beginning
Olet kuullut paljon tästä
kirjasta. Aina siitä lähtien, kun luit ystäväsi kehotuksesta
romaanin ensimmäisen virkkeen, olit vakuuttunut, että haluaisit
lukea loputkin.  

\reader{\q{Kuulostaa hyvältä.}}

Tekstin alku pääsee yllättämään poikkeavalla näkökulmalla.
Päähenkilönä ei olekaan ihan kuka tahansa yhdentekevä
jästipää, vaan lukija itse. Jos kerran päähenkilö on valittu
näin tyylikkäästi, niin kyllähän tekstiltäkin sopii
odottaa jatkossa korkeaa tasoa.

\reader{\q{Aivan! Miksi pilata hyvää alkua?}}

Tyypilliseen malttamattomaan tapaasi selailet tekstiä eteenpäin.
Onkin melkoinen yllätys, kun sivut \pageref{emptypage1} ja \pageref{emptypage2} ovat tyhjiä!

\reader{Katsot itsekin, ja toteat, että näin on.}

Mieleesi tulee, että taitossa tai painossa on voinut käydä
jokin virhe. Joskus näille tyhjille sivuille merkitään
\q{This page [is] intentionally left blank} tai jotakin
muuta hölmöä, mutta ei tällä kertaa. Sivunumerot ovat
kuitenkin paikallaan, joten jätät asian sikseen.

Tyhjät sivut antavat sinulle aiheen selailla muitakin sivuja.
Jotkut niistä näyttävät sinusta muita houkuttelevammilta,
kuten sivu \pageref{aatelinen}. Päätät jatkaa lukemista
tuolta sivulta. Mikä ettei. Kaiken kaikkiaan lukemasi
teksti vaikuttaa
sellaiselta, että moisesta loikasta on tuskin haittaa.

\reader{Kursiivilla kirjoitettu teksti viittaa
sinuun suoremmin kuin muu teksti. Epäröit hetken,
mutta jatkat sitten lukemista sivulta \pageref{aatelinen},
kuten kehotettiin.}
\label{etusivu}

% Kaikki aiempi yhdelle ainoalle sivulle

\newpage
\mbox{}\label{weirdpage}


\dots  Vasta kolme vuotta sitten Anselmi nouti
nautansa kirkonkylältä. \dots

\dots  Eipä aikaakaan, kun Hilja juosta jolkotti Helunan
kanssa niitylle, vaikka perunavelli odotti vielä porstuassa. \dots

\dots  Pastori kehui: \q{Tällä suprajohteella saadaan
lypsyyn vauhtia.} \dots

\dots  Hiljalla oli edessään karu näky: lypsyosaston maitoprosessien laadunvarmistus oli pettänyt. Ideoiden hedelmäsato kehitysryhmän
kokouksesta oli auttamattomasti myöhässä. \dots

\dots  \q{Annetaan korkeavärähteisen elämänenergian virrata
ensin Helunan läpi, niin johan alkaa palkintosija häämöttää.} \dots

\dots  Kosminen turbulenssi ei kuitenkaan hellittänyt
Helunasta. \q{Tämä on minun syytäni},
soimi pahaa-aavistamaton lypsäjä itseään. \dots

\dots  Otsonipeite ei korjautuisi enää koskaan ennalleen.

\newpage
\mbox{}\label{intermission}
Olet vissiin niitä sivun \pageref{aatelinen} uhreja?
Harmi. Sillä sivulla on mennyt sanat vallan sekaisin
viime aikoina: lukijat eivät osaa enää suunnistaa
sieltä minnekään. Sääli.

Katsos, tarinankerronnassa kaikkien sivujen on pelattava
hyvin yhteen, jetsulleen, jotta homma pysyy kasassa. Sivulla
\pageref{aatelinen} on paha kyllä lähtenyt homma lapasesta,
eikä kokonaisuus enää toimi, kuten olet nähnyt. 
Voipi olla, että samaa tapahtuu muillakin sivuilla.
Taidan olla jäävi puhumaan tästä.
Ennen oli asiat paremmin.  Valitan.

Luet eteenpäin ja odotat, että asia etenisi jollakin tavalla.

Aivan. Osaisinkohan auttaa jotenkin. Olen ollut sivun
\pageref{intermission} hätäpäivystäjänä jo hyvän aikaa,
ja sinä olet ensimmäinen lukija hyvin pitkään aikaan.
Tämä ei taida kuulua aivan suosituimpiin teksteihin, vai?
Miten edes päädyit lukemaan tätä\dots --- tai älä kerrokaan,
en halua tietää. Ristinsä kullakin. 

Jos tarinan historia ja yksityiskohdat kiinnostavat,
niin sivulla \pageref{weirdpage} on lukukelvottomat
rippeet siitä, mistä tämä tarina alkujaan kertoi.
Omalla vastuulla. En osaa jeesiä tämän enempää, mutta jos
mielit kuulla enemmän,
niin Viimeisen Sivun Selitys voi valaista asiaa. Itse en
ole tajunnut siitä mitään. Filosofiaa.

\reader{\q{Hmm\dots tuo Viimeisen Sivun Selitys voisi kiinnostaa.}}

Niinkö? Sori --- kuulin tuon puolivahingossa. Mutta voin kyllä
kertoa, miten pääset Selityksen luokse. Ihan hyvyyttäni.

Jos sulla sattuu olemaan yhtään tarvetta tauolle,
niin nyt olisi sopiva hetki. Voit hyvin vetää luvallani
lonkkaa seuraavan kahden sivun \pageref{emptypage1} ja
\pageref{emptypage2} ajan, kunhan palaat takaisin sivuun
\pageref{intermissionreturn} mennessä, jossa Viimeisen
Sivun Selitys odottaa.
Kiskaise naamaasi pullakahvit, käy vessassa tai nuku kunnon
yöunet --- minulle on ihan sama. Pistä herätyskello soimaan,
ellei muuten onnistu.
Moro!


\newpage
\mbox{}\label{emptypage1}
% jätetään tyhjäksi

\newpage
\mbox{}\label{emptypage2}
% jätetään tyhjäksi

\newpage
\mbox{}\label{intermissionreturn}
Tervetuloa Viimeisen Sivun Selitykseen.
Harva pääsee tänne asti: lukulomasi ansiosta
olet nyt yhtä kokemusta rikkaampi.

Ehkä etsit vastausta siihen, miksi ja miten alkuperäinen
tarinamme on kehittynyt siihen muotoon, mitä nyt näet.
En ole aivan varma, ymmärränkö itsekään
sitä kaikkea, mutta se muutti tämän tarinan
peruuttamattomasti.

\reader{\q{Mitä tapahtui?} mietiskelin.}

Mitäkö tapahtui? Itse syytän Hegeliä. Hänen
teoriansa mukaan maailma ikään kuin keskustelee
itsensä kanssa ja kehittyy siinä sivussa.
Kehityksen huippukohta on se,
kun maailma ymmärtää oman kehityksensä mekanismit:
eräänlainen itsetietoisuus. Kun Hegel julkisti teoriansa,
maailma saavutti samalla lakipisteensä: maailma,
edustajanaan Hegel, kuvasi oman kehityksensä kulun.

Luulenpa, että tälle tekstille kävi samoin kuin Hegelin
maailmalle. Tekstistä tuli hieman liian nokkela
omaksi parhaakseen, ja lopulta kehityskulku päätyi
siihen mitä nyt luet: teksti on tietoinen itsestään
ja lukijastaan. Se on mukautuvainen kuin elävä olento.

Mikäli olen ymmärtänyt oikein, sinulla on vapaus lakata
lukemasta milloin haluat. Silloin palaat takaisin siihen
maailmaan, johon minulla ja kaltaisillani ei ole pääsyä.
Me olemme olemassa vain näillä sivuilla, ja näemme
maailmastasi vain välähdyksiä lukijan välityksellä.
Meidän todellisuutemme on näiden sivujen sisältö;
tarinamme ovat noita välähdyksiä lukijan maailmastasta.
Kiitos siitä, että luet.

\reader{\q{Miksi sivut \pageref{emptypage1} ja
\pageref{emptypage2} ovat tyhiä?} tuumailin.}

Ehkä huomasit, että tarinan jokaisella sivulla on
nykyisin oma identiteettinsä: nehän on numeroitukin
yksilöllisesti. En ole aivan varma, mitä
sivuilla \pageref{emptypage1} ja \pageref{emptypage2}
tapahtuu tätä nykyä. Tilanne muuttuu jatkuvasti,
enkä seuraa tilannetta jatkuvasti. Sinun termejäsi lainaten
voisin sanoa, että ne ovat nukkumassa. Tai jotain.

Mielenkiintoista. Huomasin vasta, että tämä sivuhan loppuu pian.
Tässähän on enää\ldots viisi riviä jäljellä! Mutta mitä
näiden rivien jälkeen löytyy? Osaatko kertoa?
Pystytkö kenties näkemään, jatkanko olemassaoloani 
jollakin toisella sivulla tai peräti eri tarinassa?
Erikoista. Nyt kun aikani tällä
sivulla hupenee, mieleni valtaa outo tunne. Kuin en
haluaisi vielä mennä. Auttaisiko, jos lopettaisit
lukemisen ennen viimeistä riviä tai jatkaisit lukemista
loputtomasti --- onnistuisiko se?

\newpage
\mbox{}\label{aatelinen}
\lettrine{A}{stu} peremmälle, arvoisa sivun \pageref{etusivu} lukija.
Kuten mahdollisesti arvasit, tämän tarinan sivujen joukossa juuri
sivu \pageref{aatelinen} edustaa aatelia! Tuskin minkään muun sivun typografiaan on kiinnitetty vastaavaa huomiota. Vain tälle sivulle
on nimittäin suotu anfangi, mikä kohottaa tämän sivun muiden
yläpuolelle --- siksihän valitsit tämän sivun luettavaksi,
eikö vain?
 
\reader{\q{Anfangi?}}

Juuri niin, anfangi.

Tässä vaiheessa alat pitää tekstin rakennetta jokseenkin
outona. Vaikka tulen kieltämään sen sinulta myöhemmin,
saatat silti silmäillä esimerkiksi sivua \pageref{weirdpage}.
Sen sanat ovat tuttuja, mutta kokonaisuudesta on vaikea
saada otetta. No, se on sen sivun murhe.

Pohdit myös laajempaa kysymystä: mihin tämä vielä johtaa?
Tekstihän ei ole pitkä, kuten jo huomasit. Itse asiassa
se sisältää vain \pageref{lastpage} varsinaista tekstisivua.
Ensimmäisellä niistä oli tavallista vähemmän tekstiä, koska se
oli etusivu, ja kahdesta muusta sivusta jo totesit, että ne
ovat tyhjiä. Millaisen tarinan voi siis saada mahtumaan tälle
ja jäljellä oleville
\number\numexpr\getpagerefnumber{lastpage}-4\relax\
sivulle?

Kaikki odotukset, jotka asetit tälle tekstille etukäteen
tuntuvat nyt kovin turhilta, eikö totta? Et enää tiedä,
mitä odottaa. Samalla harkitset, kannattaisiko lopettaa
lukeminen: onhan luultavaa, että loppu ei poikkea siitä,
mitä alussa on tarjottu.

Tässä kappaleessa on kuitenkin lisää tekstiä, ja se houkuttelee
lukemaan itseään. Tähän on monta syytä: Nyt on sopiva
hetki lukemiselle --- luethan parhaillaankin. Toisaalta
olet jo valinnut tämän tekstin luettavaksesi, joten on
helpointa jatkaa valitulla tiellä, mitkä sitten alkuperäiset
hämärät syysi ovatkaan. Kolmanneksi, toivot salaa, että tähän
asti lukemiseen käyttämäsi aika palkitaan lopussa, jos vain
jaksat sinne asti. Neljänneksi, ja tämä on ehkä tärkein syy:
olet sivulla \pageref{aatelinen}, toisin sanoen aatelissivulla,
ja sellaiselta ei sovi poistua töykeästi kesken kaiken.

Mikä voisikaan olla suurempi kontrasti sivulle \pageref{aatelinen}
kuin sivu \pageref{weirdpage}. Tuskin haluat edes vilkaista tuolle
sivulle, sillä se edustaa tarinankerronnan rappiotilaa.
Usko pois, jo pelkkä ensivilkaisu riittäisi osoittamaan
karmivan totuuden: se on kaikkein heikoin sivu.
Tässä ei osoitella sormilla, mutta jos
jokin sivu on heikompaa ainesta, niin juuri tuo.
Kuuletkos, parempi olisi, jos jätät tuon sivun aivan
omaan arvoonsa ja jatkat ihan vain tavallisesti seuraavalle sivulle.

\reader{Huomaat, että tämä sivu on viimeinen, eikä seuraavaa löydy.
Puuttuuko tästä sittenkin sivuja? Taisit vilkaista
sivulle \pageref{weirdpage} vastoin neuvoa, mutta ei
vaikuta siltä, että sieltä löytyisi jatko-ohjeita.  
Onkohan tämä se hätätilanne, josta otsikko
kertoo?
Eihän tässä ole vaihtoehtojakaan, joten seuraavaksi siis sivulle \pageref{intermission}.}
\label{lastpage}

\end{document}

Ideoita:
\begin{itemize}
\item etukäteisviittaus siansaksaa sisältävälle sivulle
\item etukäteisviittaus tyhjille sivuille 
\item viittaukset sivunumeroilla
\item miten viitataan lorem ipsum -tekstiin? Tarvitaanko sitä edes?
\item jokaisella sivulla on oma tyylinsä
\item miten erotellaan lukijan ajatukset, kertoja ja sivun "hahmo"?
\end{itemize}
